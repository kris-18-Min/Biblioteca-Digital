\
\documentclass{article}
\usepackage[utf8]{inputenc}
\usepackage{hyperref}
\usepackage{listings}
\usepackage{graphicx}
\title{Desarrollo de Proyecto con Arquitectura Hexagonal y RabbitMQ}
\author{Equipo — Biblioteca Digital}
\date{\today}

\begin{document}
\maketitle

\section*{Resumen}
Este documento describe el proyecto: sistema de gestión de biblioteca digital usando arquitectura hexagonal, Node.js, React + Vite y RabbitMQ.

\section{Objetivos}
\begin{itemize}
\item Aplicar la arquitectura hexagonal separando dominio, puertos y adaptadores.
\item Implementar mensajería con RabbitMQ entre microservicios.
\item Desarrollar un frontend moderno con React + Vite.
\item Documentar todo en Overleaf con evidencia técnica.
\end{itemize}

\section{Arquitectura}
Se presentan dos microservicios principales: \texttt{users-service} (gestión de usuarios y libros) y \texttt{admin-service} (auditoría). RabbitMQ actúa como bus de eventos.

\section{Evidencia técnica}
Incluya capturas de pantalla del frontend, logs de auditoría y fragmentos de código. Por ejemplo, endpoint de registro:
\begin{lstlisting}[language=JavaScript]
POST /users
body: { name, email, password }
\end{lstlisting}

\section{Reflexión}
Describir decisiones de diseño, retos y oportunidades de mejora (por ejemplo: testing, despliegue continuo, escalado).

\section{Conclusión}
Resumen y aprendizajes.

\end{document}
